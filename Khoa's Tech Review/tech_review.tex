\documentclass[onecolumn, draftclsnofoot,10pt, compsoc]{IEEEtran}
\usepackage{graphicx}
\usepackage{url}
\usepackage{setspace}

\usepackage{geometry}
\geometry{textheight=9.5in, textwidth=7in}

% 1. Fill in these details
\def \CapstoneTeamName{		    BBM}
\def \CapstoneTeamNumber{		26}
% \def \GroupMemberOne{			Khoa Tran}
% \def \GroupMemberTwo{			Aaron Leensknecht}
% \def \GroupMemberThree{		Karl Popper}
\def \CapstoneProjectName{		Baby Body Measurement Using Computer Vision}
\def \CapstoneSponsorCompany{	School of Electrical Engineering and Computer Science}
\def \CapstoneSponsorPerson{	D. Kevin McGrath}

% 2. Uncomment the appropriate line below so that the document type works
\def \DocType{		Problem Statement
				%Requirements Document
				%Technology Review
				%Design Document
				%Progress Report
				}
			
\newcommand{\NameSigPair}[1]{\par
\makebox[2.75in][r]{#1} \hfil 	\makebox[3.25in]{\makebox[2.25in]{\hrulefill} \hfill		\makebox[.75in]{\hrulefill}}
\par\vspace{-12pt} \textit{\tiny\noindent
\makebox[2.75in]{} \hfil		\makebox[3.25in]{\makebox[2.25in][r]{Signature} \hfill	\makebox[.75in][r]{Date}}}}
% 3. If the document is not to be signed, uncomment the RENEWcommand below
\renewcommand{\NameSigPair}[1]{#1}

%%%%%%%%%%%%%%%%%%%%%%%%%%%%%%%%%%%%%%%
\begin{document}
\begin{titlepage}
    \pagenumbering{gobble}
    \begin{singlespace}
    	\includegraphics[height=4cm]{coe_v_spot1}
        \hfill 
        % 4. If you have a logo, use this includegraphics command to put it on the coversheet.
        %\includegraphics[height=4cm]{CompanyLogo}   
        \par\vspace{.2in}
        \centering
        \scshape{
            \huge CS Capstone \DocType \par
            {\large\today}\par
            \vspace{.5in}
            \textbf{\Huge\CapstoneProjectName}\par
            \vfill
            {\large Prepared for}\par
            \Huge \CapstoneSponsorCompany\par
            \vspace{5pt}
            {\Large\NameSigPair{\CapstoneSponsorPerson}\par}
            {\large Prepared by }\par
            Group\CapstoneTeamNumber\par
            % 5. comment out the line below this one if you do not wish to name your team
            \CapstoneTeamName\par 
            \vspace{5pt}
            {\Large
                \NameSigPair{\GroupMemberOne}\par
                %\NameSigPair{\GroupMemberTwo}\par
                %\NameSigPair{\GroupMemberThree}\par
            }
            \vspace{20pt}
        }
        \begin{abstract}
        % 6. Fill in your abstract    
        	%This document is written using one sentence per line.
        	%This allows you to have sensible diffs when you use \LaTeX with version control, as well as giving a quick visual test to see if sentences are too short/long.
        	%If you have questions, ``The Not So Short Guide to LaTeX'' is a great resource (\url{https://tobi.oetiker.ch/lshort/lshort.pdf})%
        	
        	This document introduces the four main components that comprise this project, and brings into focus a few sub-components in order to explore various technologies.
        	Specifically, it examines the differences between iOS Augmented Reality Kit versions 2 and 3, mobile development frameworks, and libraries to draw user-friendly charts for the user.
        \end{abstract}     
    \end{singlespace}
\end{titlepage}
\newpage
\pagenumbering{arabic}
\tableofcontents
% 7. uncomment this (if applicable). Consider adding a page break.
%\listoffigures
%\listoftables
\clearpage

% 8. now you write!
\section{Introduction}
As is typical for a mobile application, the product consists of two main layers: the back end and the front end.
The back end handles the core functionality of the application, which is to measure a baby's body using computer vision.
The front end manages the user interface and interactions. There are four main components to this application.
The back end consists of the \textit{computer vision (CV)} framework and \textit{data storage} (local and cloud).
The front end consists of the \textit{user interface}. Common to both layers is the \textit{integrated development environment (IDE)}.
\\\\
\noindent Except for IDE, each main component is further broken down into sub-components:
\begin{itemize}
    \item Computer vision framework (e.g. ARKit, ARCore):
    \begin{itemize}
        \item Head circumference estimation
        \item Height estimation
    \end{itemize}
    
    \item User interface:
        \begin{itemize}
            \item Mobile application framework (e.g. SwiftUI, React Native, Xamarin)
            \item Charting library available to a framework
        \end{itemize}
    
    \item Data storage:
        \begin{itemize}
            \item Local database management system (DBMS) (e.g. sqlite)
            \item Cloud hosting services (Amazon Web Services, Google Cloud Platform, Azure)
        \end{itemize}
\end{itemize}

\noindent This technology review focuses on the differences between versions of Augmented Reality Kit (ARKit) framework (desired by our client), mobile application framework, and charting library.


\section{ARKit Versions}
While the application doesn't exactly involve augmented reality (AR), it can take advantage of AR frameworks to get measurements from objects in an environment.
From our initial research into the latest version of ARKit (version 3), it was revealed that the framework supports pose estimation in particular, which is helpful to estimate the baby's height.

\subsection{Version 1}
ARKit was first featured in iOS 11, released in September 2017. An iOS device with an A9 chip was required to support it \cite{ARKit_A9_processor}.
This meant augmented reality capability was restricted to the iPhone 6S and 5th generation iPad or later.
\\\\
The framework had a couple of features of interest: plane estimation with basic boundaries and scale estimation \cite{ARKit1_features}.
If a top-down view of a baby's head can be interpreted as a circle, from which the application can acquire the diameter by measuring from edge to edge, then a basic plane estimation will give us a rough measurement of the circumference from applying formula of the circle's circumference.
\\\\
On the other hand, the application can use scale estimation to approximate the body length. Of course, the baby's limbs need to be straightened. However, one of the project's goals is to ease the measuring process on the baby and the people involved; having to handle the baby partially defeats the purpose.

\subsection{Version 2}
ARKit version 2 came with the release of iOS 12 in September 2018.
It did not offer new features of interest. However, iOS was bundled with the Measure app \cite{measure_app}, a tape measuring tool.
Using a white dot in the center of the screen as a "pinning tool", the user can pin a pair of reference points across which they want to get the length.
The application can incorporate this functionality to get the radius (or diameter) of a baby's head without a top down view, as the user can simply place reference points at the sides of the forehead.
However, a top-down view still produces a better estimation, as the top of the head is flatter than the forehead.
\\\\
Additionally, the functionality can be used to measure height from top of the head to feet. Bent limbs can be worked around by placing reference points at the joints.
Afterwards, the user can calculate the sum of partial lengths from the joints to get the height. This is prone to user error, however, but it solves the issue of bent limbs.

\subsection{Version 3}
ARKit version 3 came as part of the release of iOS 13 in September 2019. The updated framework came with two new features: modeling a skeleton of the body (pose estimation) \cite{ARBody2D} and automatically estimate the skeleton's height. \cite{automaticSkeletonScaleEstimationEnabled}.
\\\\
Mapping a pose estimation onto a baby's body automates the process of pinning reference points. Furthermore, the automatic height estimation does the job for the app entirely.
However, the method needs to be tested for accuracy.
Most importantly, it has to be able to measure height with bent limbs.

\subsection{Discussion}
A rather unfortunate restriction of ARKit 3 is it is only supported on iOS devices with the A12 processors \cite{ARKit3_A12_processor}.
This means if the application used these built-in features, it would only work on the iPhone XS and 11-inch and 12.9-inch 2018 iPad Pros or later.
This high bar of entry greatly narrows our consumer base. As of December 2018, only 5.48\% of iPhones are the XS and XS Max \cite{iPhone_percentage}, due to the iPhone X brand's low sales numbers.

\subsection{Conclusion}
This section explores the difference in versions of ARKit. The team actually are debating whether to use ARKit, thereby limiting our application to iOS devices, or another cross-platform framework like ARCore or OpenCV. This comparison is in fact done on another team member's tech review. The choice of version in this case assumes the application uses ARKit.
\\\\
Accessibility for our users strongly favors ARKit 2, though it means the application misses out on precise measurement tools like skeleton modeling and height estimation.
If the application emulates the Measure app featured in iOS 12 (developed with ARKit 2), the user can place reference points at the joints for limbs that are bent.
However, our client desires simplicity---he wants to just snap a picture of a baby and see the app report its measurement. Therefore, the team will have to develop a method for pose estimation.


\section{Mobile Application Framework}
The mobile application framework will be used to develop a user interface. Such framework typically provides a model, view, and controller (MVC) structure. The model interacts with the back end of the application, such initiating estimation methods, and passes back the resulting data. The view is the literal graphical interface presented to the user. Lastly, the controller handles user input and passes it to the model.

\subsection{SwiftUI}
SwiftUI is exclusive to Apple platforms. The framework is native, which means the end product runs natively on the iOS device. This increases the efficiency and speed of the application.
Its syntax is derived from the declarative programming paradigm in which the developer can state the desired outcome for the user interface. The implementation is done behind the scenes \cite{swiftui}.
\\\\
The framework is supported on Xcode, the IDE exclusive to macOS.The IDE supports both what-you-see-is-what-you-get (WYSIWYG) (i.e. a design canvas) and and what-you-see-is-what-you-mean (WYSIWYM) (i.e. a code editor) editors for polishing UI. When one editor changes (e.g. a button is styled with rounded corners), the other is immediately updated with the corresponding change \cite{swiftui}.
This makes it very easy for the developers to always have an idea of what they're working with.

\subsection{Ionic}
Ionic is an open-source cross-platform framework. It is considered a hybrid framework, which means code is wrapped by a thin client that translates framework code into native code. There is some performance and efficiency cost by definition.
\\\\
It leverages web technologies (HTML, CSS, and JavaScript) to build performant and high-quality mobile apps. This lowers the barrier to entry, as it is typical for an mobile developer to have some experience with web technologies. For advanced developers, Ionic also provides integrations with JavaScript frameworks like Angular and React. Integration with Vue is in development \cite{ionic}.
\\\\
Because of its open-source nature, the financial barrier is non-existent, aside from developer resource. Anyone can use it to deploy a mobile app to Android or iOS.

\subsection{Xamarin}
Developed by Microsoft, Xamarin is a cross-platform hybrid framework based on the .NET platform.
It uses C\# as the programming language.
However, the framework provides facilities for accessing third-party code written in Objective-C (the predecessor of Swift for iOS app development) and Java (Android app development) \cite{xamarin}.
\\\\
Usage of the framework requires Visual Studio which is exclusive to Windows. However, aside from the features typical of an IDE (code completion, source control, etc.), Visual Studio does not offer features specific to app development.

\subsection{Discussion}
The first and foremost decision the team need to make is whether to develop an iOS-exclusive or cross-platform application. If the application is cross-platform, then SwiftUI is out of the question. Though SwiftUI has an advantage in native performance (though it may not be noticeable for an app of this scale), its exclusive ties to Xcode and macOS bring a hidden cost to an otherwise free framework.
\\\\
Xamarin is on the other hand exclusive to Visual Studio which in turn only runs on Windows. Additionally, it uses technologies unfamiliar to the team (C\#, .NET, and Model-View-ViewModel structure), which would take time to learn.
\\\\
While our client's preference is a native ARKit application, he has given us the license to develop cross-platform as long as the iOS version works well. Compared to the other frameworks, Ionic is much more accessible in many ways. Firstly, neither macOS nor Windows is required for development. Secondly, it uses web technologies, which the team has more exposure to compared to Swift or C\#.


\section{Charting Library}
A chart, such as a bar or line chart, provides the user with a visual view of the baby's growth. The chart also gives the option for the user to toggle a side-by-side or overlaid comparison to a standard growth chart from the World Health Organization (WHO) or United States Center for Disease Control (CDC).
\\\\
From the team's survey, only Ionic and Xamarin provide libraries for making charts. While developing charts in SwiftUI is possible, a solution needs to be built from scratch.

\subsection{Chart.js (Ionic)}
Chart.js is an open-source library provided by the developer community. It supports many types of charts, including bar and line charts \cite{chart.js}.
\\\\
It's purportedly very easy to use. Once imported as a component, a chart is invoked using a single \textless canvas\textgreater tag as part of an HTML file. It can be configured visually by passing certain attributes in the corresponding JavaScript file (e.g. type, labels, data sets, colors) \cite{chart.js_intro}.
\\\\
Additionally, it provides useful user interaction events such as tooltip popups and trackball. The latter in particular is helpful because the user can touch and trace their finger along a line to continually reveal growth measurements.

\subsection{Xamarin.Forms Charts \& Graphs}
Xamarin.Forms library, provided by a third-party company named SyncFusion, provides charting capability. Similar to Chart.js, it supports bar and line charts and various configuration options \cite{xamarin_charts}.
\\\\
A chart can be inserted into the view with the written in XAML (Xamarin Markup Language) similarly to HTML. Configurations are set in C\#, similarly to JavaScript.

\subsection{Discussion}
Chart.js and Xamarin.Forms Charts \& Graphs are very similar. Of course, the latter is part of Xamarin; thus, using it requires Windows and Visual Studio.
\\\\
Chart.js, like Ionic, can be used on any platform.

\subsection{Conclusion}
In line with the team's favor of the Ionic framework for accessibility and ease of use (elaborated on in the previous section), Chart.js is the better choice.


\section{Summary}
After conducting a tech review into the computer vision framework, mobile application framework, and charting library, the team has chosen these three technologies:
\begin{itemize}
    \item ARKit version 2 (if ARKit is adopted)
    \item Ionic mobile application framework
    \item Chart.js (Ionic library)
\end{itemize}
    
\noindent The capability of the tools in each category is neck-and-neck in comparison. Our choices ultimately came down to accessibility (whether the platform is locked into an exclusive operating system), cost, and familiarity with the programming languages.

\bibliographystyle{IEEEtran}
\bibliography{tech_review.bib}

\end{document}