\documentclass[letterpaper,10pt,draftclsnofoot,onecolumn,compsoc]{IEEEtran}
\usepackage{graphicx}
\usepackage{amssymb}
\usepackage{amsmath}
\usepackage{array}
\usepackage{amsthm}
\usepackage{listings}
\usepackage{alltt}
\usepackage{float}
\usepackage{color}
\usepackage{tabularx}
\usepackage{url}
%\usepackage{titlesec}
\usepackage{setspace}
\usepackage{rotating}
\usepackage{lscape}
\usepackage{balance}
\usepackage{enumitem}
\usepackage{graphicx}
\usepackage{float}
%\usepackage{pstricks, pst-node}
\usepackage{inputenc}
\usepackage[margin=.75in]{geometry}
\usepackage[section]{placeins}

\newcommand{\subparagraph}{}
\usepackage{titlesec}
\setlength{\abovecaptionskip}{2pt}
\setlength{\belowcaptionskip}{0pt}
\usepackage{fancyhdr}
\usepackage{hyperref}
\usepackage{tocloft}

\lstset{language=HTML,
        showstringspaces=false}

%hide toc subsubsections
\setcounter{tocdepth}{2}
\setlength{\parindent}{.0in}

%toc formatting for IEEE 830-1998 standards
\renewcommand{\cftsecleader}{\cftdotfill{\cftdotsep}{\vspace{.25cm}}}
\renewcommand{\cftsecfont}{\normalfont}
\renewcommand{\cftsecpagefont}{\normalfont}
\renewcommand{\cftsecaftersnum}{.}

%bottom right page numbers
\fancyhf{}
\renewcommand{\headrulewidth}{0pt}
\rfoot{\thepage}
\pagestyle{fancy}

%formatting specific IEEE 830-1998 Section headings
\titleformat{\section}[block]
  {\fontsize{12}{15}\bfseries\sffamily}
  {\thesection.}
  {1em}
  {}
\titleformat{\subsection}[block]
  {\fontsize{10}{10}\bfseries\sffamily}
  {\thesubsection}
  {1em}
  {\vspace{.1cm}}
\titleformat{\subsubsection}[block]
  {\fontsize{10}{10}\bfseries\sffamily}
  {\thesubsubsection}
  {1em}
  {\vspace{.2cm}}
\geometry{textheight=8.5in, textwidth=6in}

\newcommand{\cred}[1]{{\color{red}#1}}
\newcommand{\cblue}[1]{{\color{blue}#1}}

\def\name{Aidan Grimshaw}

%% The following metadata will show up in the PDF properties
\hypersetup{
  urlcolor = black,
  pdfauthor = {\name},
  pdfkeywords = {cs461 ``Senior Capstone - Fall 2018'' capstone},
  pdftitle = {CS 461 Tech Review},
  pdfsubject = {Capstone Tech Review},
  pdfpagemode = UseNone
}

\begin{document}
\begin{titlepage}
\centering
\vspace*{6cm}
{\scshape\LARGE \begin{singlespace}Baby body measurement using computer vision\\ \end{singlespace} Team 26: BBM \\ Tech Review } \\
	{\scshape\Large CS461 - Fall 2019 \par}
	\vspace{.5cm}
	\name \par
    {\large \today \par} 
	\vspace*{1cm}
	
\end{titlepage}

\section{Project Role}
My role in the project is to collaborate on implementing the measurement functionality for the Baby subjects as well as creating a persistent data store for the results of those measurements in a structured DB format.

\section{What your team is trying to accomplish}

The problem is simply that the manual measurement process for a child is labor intensive and can be improved. The solution is leveraging computer vision to automatically get all needed measurements from a single picture of the child. This will enable parents to get this information quicker, cheaper, and in areas where medical providers are not easily available.
% What your three pieces or personal responsibilities for this project are at a more detailed level...(piece 1, piece 2, piece 3). Alternatively, you might write about experimental methods, processes, study design, or something else, depending on the topic of your Capstone Project. 
\section{Responsibilities}
I will be researching 3 different logical components that must be implemented: head estimation, height estimation, and an on phone local database. 

\subsection{Head Circumference Estimation} 
% (arkit vs external), 

There are roughly two ways that head circumference can be measured using a mobile phone camera.\\

\subsubsection{Choices}
1. A measurement can be created using reference object with known dimensions. The reference dimensions are used to build a mapping from number of pixels to distance. From this mapping, other objects in the image can be measured, assuming that they have the same depth from the camera as the reference object. This can be done using image processing frameworks like OpenCV\\

2. A measurement can be created by using video frames and motion sensors to find the dimensions of an object. Using the motion sensors of the mobile phone, when the phone moves, the phone can tell roughly the distance it moved. Using the image data from the starting point of the movement and the ending point of the movement, the phone software can triangulate the distance to the object from the phone, and from that get the dimensions of the object. This is can be done trivially with native IOS frameworks like ARKit, or by Google's alternative, ARCore. These frameworks allow the user to pick points from the phone's camera and derive distances between those points.\\

Using the core measurement from one side of the subjects head to the other side, the diameter, head circumference can be derived by multiplying the diameter by \pi.

\subsection{Height Estimation}
% /body pose estimation (arkit vs external), 
Height estimation can be done using the same basic building blocks as head circumference. However, there is an additional complication. Since the subject of the measurement may not necessarily be standing straight, calves, thighs, torso and head must be measured individually, and their values combined to get the overall height of the subject.\\

In order to accomplish this task we will employ pose estimation technology to get the starting and ending points for all of the sections of the body outlined above. Pose estimation finds the joints of a person figure in an image. Using the points gathered from the pose estimation, we can combine the points with native measurement frameworks like ARKit or ARCore to get all distances between the points picked out, and from there, body length.\\

Arkit has pose estimation built into the framework, and doesn't require any external dependencies. Arcore doesn't come with pose estimation functionality built in, and must be combined with any number of open source pose estimation models online. We may try it in conjunction with PoseNet from TensorFlow as a good starting place.

\subsection{Local Database}
% local database.
In order to track the growth progress of the subject, a persistent data store must be used to keep track of historical data. We have decided to implement this in the form of a local database on the phone because it minimizes any cost load we would have storing the data, would make the service accessible to users even without internet, and made sure that the user data stayed private if they want it to be.\\

SQLite is the most established and widely used mobile database, however alternatives include Couchbase Lite and ForestDB. 

% List possible technologies, methods, or options for each piece or part that you're managing for the project and examine these in more detail; (identify other potential technologies even if your client has told you which one to use). You'll need to consult and cite sources of information to complete this final task. 



\end{document}
