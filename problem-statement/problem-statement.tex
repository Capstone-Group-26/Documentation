\documentclass[10pt, draftclsnofoot, letterpaper, margin=.75in, onecolumn]{IEEEtran}
% \IEEEoverridecommandlockouts
% The preceding line is only needed to identify funding in the first footnote. If that is unneeded, please comment it out.
\usepackage{cite}
\usepackage[margin=0.75in]{geometry}
\usepackage[singlespacing]{setspace}
\usepackage{amsmath,amssymb,amsfonts}
\def\BibTeX{{\rm B\kern-.05em{\sc i\kern-.025em b}\kern-.08em
    T\kern-.1667em\lower.7ex\hbox{E}\kern-.125emX}}
\begin{document}

\renewcommand{\familydefault}{\sfdefault}

\title{Problem Statement}
\author{\IEEEauthorblockN{Aidan Grimshaw, Aaron Leenknecht, Khoa Tran\\}
\IEEEauthorblockA{\textit{Fall 2019 CS461: Senior Design\\}
\textit{Oregon State University\\}}
Computer Vision For Measuring Infants}


\begin{titlepage}
\maketitle
\begin{abstract}
\noindent This document will describe the problem that is currently being solved, the rationale behind why the problem should be solved, the proposed solution, and performance metrics for the project. The problem is simply that the manual measurement process for a child is labor intensive and can be improved. The solution is leveraging computer vision to automatically get all needed measurements from a single picture of the child. This will enable parents to get this information quicker, cheaper, and in areas where medical providers are not easily available.
\end{abstract}

\end{titlepage}

%\begin{IEEEkeywords}
%component, formatting, style, styling, insert
%\end{IEEEkeywords}

\section{Problem Definition}
\par \noindent Currently, when a baby is taken to the doctor for routine checkups, part of the checkup process is measuring the child's height and width, as well as several other measurements. This process can be difficult with infants: they squirm around which makes it labor intensive for the parents or doctors to accurately measure the body. The process may be able to be done faster at home by parents if it is done through a computer vision application rather than by hand. \\

\par \noindent After the application obtains the photo of the baby, it must calculate limb length, head circumference, and overall body length. The measurements need to put in context to provide useful information to the user. Users must then be relayed correct information on next steps to take based on what their specific measurements entail.\\

\par \noindent Although it may seem unintuitive, building an application that addresses these issues also may improve several other healthcare factors, including cost, early disease screening, and health service accessibility.\\

\textbf{Cost}
\par \noindent It is often difficult for worried parents to tell whether their baby is developing healthily and normally. This can lead to extra hospital visits which increases healthcare costs for all healthcare consumers. The American Academy of Pediatrics (AAP) recommends babies get checkups at birth, 3 to 5 days after birth, and at 1, 2, 4, 6, 9, 12, 15, 18 and 24 months\cite{checkup}. Cost of these checkups average around 100 dollars per visit\cite{cost}, while the average number of new births each year in the US is around 3.8 million\cite{births}. If only 1 percent of children can be prevented from making an extra visit to the doctor's, this would conservatively save 3.8 million dollars from the healthcare system.\\

\textbf{Early Disease Screening}\\
This screening would allow parents to check if their baby is far outside the average range in height or width for their age. In rare cases, this can indicate the following:\\
\begin{itemize}
\item Congenital hypothyroidism 1 in 2,500 to 3,000 babies each year are born with this, translating to 1266 babies every year in the US.\\
\item Congenital Growth hormone deficiency 1 in 7,000 babies each year are born with this, translating to 542 babies every year in the US. \\
\end{itemize}
Earlier feedback from these diagnoses can lead to better health outcomes for infants, who would otherwise be diagnosed later and need more intensive treatment or have worse lifetime health as a result. If congenital hypothyroidism is not treated, it can cause serious problems such as mental disability, growth delays, or loss of hearing. If congenital Growth hormone deficiency is untreated, it may result in slower muscular development, reduced energy, impaired concentration and memory loss, greatly reduced height, and reduced bone mass and osteoporosis\\

\textbf{Access}\\
In rural areas or countries where immediate access to doctors and hospitals is uncommon, this application may benefit parents who would not otherwise have easy access to the checkups that pediatricians provide, allowing them access to medical services that they might not otherwise have.

\section{Proposed Solution}
\par \noindent The team is working on creating an IOS app that measures infants height, width, and possibly other measurements that can be derived from images. These measurements will be compared against WHO and NHS guidelines for healthy metrics at the current age of the child. The data will also be stored and charted against the child’s past data to make sure that they are following a healthy pattern of growth. In order to do this, we will leverage native computer vision APIs that are accessible to IOS developers as part of ARkit. We may also use a database for storing the child measurement data such as MongoDB or Firebase, and an API to be able to share those measurements with the child’s pediatrician, who will do various additional screenings if the child falls outside of the normal range of measurements for it’s age group.\\

\par \noindent This will allow parents and healthcare professionals to solve the problems outlined in the problem definition. The application will save on cost by reducing medical visits and time spent on those visits, as well as screening babies for possible diseases that can be referred to doctors for further investigation. The application will also enable users in remote areas to access these services which they would otherwise be denied. Overall, this application will provide peace of mind for most parents, confirming that their babies are developing normally, while providing some with referrals to pediatricians if they might need it. It may also enable health care providers insights into data about the size of children at different years in an easily accessible and fully digital fashion, rather than relying on paper charts and graphs\\

\textbf{Technology}
\par \noindent iOS SDK offers the ARKit (augmented reality kit) framework to developers to use on an iOS device's camera. While we're not looking to use augmented reality particularly for this project, the framework offers methods for detecting 2D objects and visualizing them in 3D; this is useful for scanning specific body parts and mapping them in 3D for measuring (e.g. head circumference). Additionally, developing for iOS means we can take advantage of Swift, an industry-adopted, modern programming language. \\

\section{Performance Metrics}
\par \noindent The team will know it has successfully completed the project if it has completed 4 key metrics for the project, accuracy, usability, cost, and compliance. Assuming that we complete these metrics, we will consider the project successfully completed.\\

\textbf{Accuracy}
\par \noindent Because the height of the child is checked against the average percentiles for it’s age, the application does not have to be exact in its measurements. However, any sufficiently large deviation would make the tool less effective at placing the child within the range and possibly lose the trust of parents. Therefore, we will target a deviation below 5 percent between predicted and actual infant measurements. The accuracy target is arguably the most crucial metric to meet because without an accurate measurement, the project will not be useful to users regardless of the completion of the other two goals.\\

\textbf{Usability}
\par \noindent The application has to enable untrained users to effectively take the measurements and view results. The team may utilize ARkit for creating on screen guidance overlays when an image is being taken, to assist parents in taking the image. The team may also utilize growth charts for the child and comparing it to the mean in its age group and a progress chart to track changes in the child's height.\\

\textbf{Cost}
\par \noindent In order to enable as many parents and care providers to use the app as possible, it will be provided for free to all end users in the app store.  The code will also be available under open source licencing on Github, to enable developers to expand and improve upon the work that we contributed to the project.\\

\textbf{Compliance}
\par \noindent We will need to discuss the legality surrounding the use of infant pictures for research, development, and testing. Mr. McGrath did not require us to sign any kind of agreement in his project proposal. While we may be able to use stock photos of babies for the project, if we’d like to use baby photos from a clinical environment, we would need the parents’ consent. It is crucial that we comply with all laws and regulations.\\

\textbf{Deliverables}
\par \noindent Our goal is to deliver a working prototype, i.e. an iOS app that's able to measure a baby's head circumference, overall body length, and individual limb lengths, by the Engineering Expo. Ideally, we'd like to nail down the following general tasks by the following deadlines:
\begin{itemize}
    \item Finalize implementation specifics -- Nov. 1
    \item Finalize core functions -- Nov. 13
    \item Develop UI concepts -- week 5 of Winter term
    \item Deliver an alpha prototype -- end of Winter term
    \item Fix bugs and implement changes from feedback -- week 2 of Spring term
    \item Finish testing and implement final changes -- week 5 of Spring term
    \item Prototype is ready for release - Engineering Expo
\end{itemize}

\bibliographystyle{IEEEtran}
\bibliography{problem-statement/problem-statement.bib}

\end{document}