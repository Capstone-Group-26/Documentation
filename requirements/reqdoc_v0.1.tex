\documentclass[letterpaper,10pt,draftclsnofoot,onecolumn,compsoc]{IEEEtran}
\usepackage{graphicx}
\usepackage{amssymb}
\usepackage{amsmath}
\usepackage{array}
\usepackage{amsthm}
\usepackage{listings}
\usepackage{alltt}
\usepackage{float}
\usepackage{color}
\usepackage{tabularx}
\usepackage{url}
%\usepackage{titlesec}
\usepackage{setspace}
\usepackage{rotating}
\usepackage{lscape}
\usepackage{balance}
\usepackage{enumitem}
\usepackage{graphicx}
\usepackage{float}
%\usepackage{pstricks, pst-node}
\usepackage{inputenc}
\usepackage[margin=.75in]{geometry}
\usepackage[section]{placeins}

\newcommand{\subparagraph}{}
\usepackage{titlesec}
\setlength{\abovecaptionskip}{2pt}
\setlength{\belowcaptionskip}{0pt}
\usepackage{fancyhdr}
\usepackage{hyperref}
\usepackage{tocloft}

\lstset{language=HTML,
        showstringspaces=false}

%hide toc subsubsections
\setcounter{tocdepth}{2}
\setlength{\parindent}{.0in}

%toc formatting for IEEE 830-1998 standards
\renewcommand{\cftsecleader}{\cftdotfill{\cftdotsep}{\vspace{.25cm}}}
\renewcommand{\cftsecfont}{\normalfont}
\renewcommand{\cftsecpagefont}{\normalfont}
\renewcommand{\cftsecaftersnum}{.}

%bottom right page numbers
\fancyhf{}
\renewcommand{\headrulewidth}{0pt}
\rfoot{\thepage}
\pagestyle{fancy}

%formatting specific IEEE 830-1998 Section headings
\titleformat{\section}[block]
  {\fontsize{12}{15}\bfseries\sffamily}
  {\thesection.}
  {1em}
  {}
\titleformat{\subsection}[block]
  {\fontsize{10}{10}\bfseries\sffamily}
  {\thesubsection}
  {1em}
  {\vspace{.1cm}}
\titleformat{\subsubsection}[block]
  {\fontsize{10}{10}\bfseries\sffamily}
  {\thesubsubsection}
  {1em}
  {\vspace{.2cm}}
\geometry{textheight=8.5in, textwidth=6in}

\newcommand{\cred}[1]{{\color{red}#1}}
\newcommand{\cblue}[1]{{\color{blue}#1}}

\def\name{Aidan Grimshaw, Aaron Leenknecht, Khoa Tran}

%% The following metadata will show up in the PDF properties
\hypersetup{
  urlcolor = black,
  pdfauthor = {\name},
  pdfkeywords = {cs461 ``Senior Capstone - Fall 2018'' capstone},
  pdftitle = {CS 461 Requirements Document},
  pdfsubject = {Capstone Requirements Document},
  pdfpagemode = UseNone
}

\begin{document}
\begin{titlepage}
\centering
\vspace*{6cm}
{\scshape\LARGE \begin{singlespace}Baby body measurement using computer vision\\ \end{singlespace} Team 26: BBM \\ Software Requirements Specification } \\
	{\scshape\Large CS461 - Fall 2019 \par}
	\vspace{.5cm}
	\name \par
    {\large \today \par} 
	\vspace*{1cm}
	
\begin{abstract}
\begin{singlespace}
This document will outline a set of functionality, performance, usability, and capability requirements for the baby body measurement application. The baby body measurement application will be an IOS app that uses computer vision to measure babies and use their measurements to derive basic health metrics. The app will be targeted at recent parents.
\end{singlespace}
\end{abstract}

\end{titlepage}

\newpage

\tableofcontents

%removes page number on table of contents
\thispagestyle{empty}

\newpage

% \section*{Revision History}
% \begin{center}
%     \begin{tabular}{|p{1cm}|p{7cm}|p{7cm}|p{2cm}|}
% 		\hline
%         PR-2 & The README documentation on GitHub shall be sufficient enough that a daily user of IDA can install this software in less than 30 minutes & The documentation present on GitHub (README) shall have installation, use, and testing instructions for Mac OSX High Sierra, Windows 10, and Ubuntu 16.04 & 1/20/19\\
%         \hline
%         CR-7 & The code, documentation (both formal and informal) shall be completely open-source and free for all & The code and documentation shall be open source & 1/20/19\\
%         \hline
%     \end{tabular}
% \end{center}

\section{Introduction}
\begin{singlespace}
\noindent
Currently, when a baby is taken to the doctor for routine checkups, part of the checkup process is measuring the child's height and width, as well as several other measurements. This project is to create an iOS application that obtains the photo of the baby and calculates head circumference and overall body length. The measurements are then related to the standard WHO growth charts to give a parent a comparison of their baby's growth.
\end{singlespace}

\subsection{Purpose}
\begin{singlespace}
\noindent

\par Although it may seem unintuitive, building an application that finds simple baby measurement metrics may improve several other healthcare factors, including cost, early disease screening, and health service accessibility.\\

\textbf{Cost}
\par It is often difficult for worried parents to tell whether their baby is developing healthily and normally. This can lead to extra hospital visits which increases healthcare costs for all healthcare consumers. The American Academy of Pediatrics (AAP) recommends babies get checkups at birth, 3 to 5 days after birth, and at 1, 2, 4, 6, 9, 12, 15, 18 and 24 months\cite{checkup}. Cost of these checkups average around 100 dollars per visit\cite{cost}, while the average number of new births each year in the US is around 3.8 million\cite{births}. If only 1 percent of children can be prevented from making an extra visit to the doctor's, this would conservatively save 3.8 million dollars from the healthcare system.\\

\textbf{Early Disease Screening}\\
\par This screening would allow parents to check if their baby is far outside the average range in height or width for their age. In rare cases, this can indicate the following:\\

\begin{itemize}
\item Congenital hypothyroidism: 1 in 2,500 to 3,000 babies each year are born with this, translating to 1266 babies every year in the US.\\
\item Congenital Growth hormone deficiency: 1 in 7,000 babies each year are born with this, translating to 542 babies every year in the US. \\
\end{itemize}

\par Earlier feedback from these diagnoses can lead to better health outcomes for infants, who would otherwise be diagnosed later and need more intensive treatment or have worse lifetime health as a result. If congenital hypothyroidism is not treated, it can cause serious problems such as mental disability, growth delays, or loss of hearing. If congenital Growth hormone deficiency is untreated, it may result in slower muscular development, reduced energy, impaired concentration and memory loss, greatly reduced height, and reduced bone mass and osteoporosis\\

\textbf{Access}\\
In rural areas or countries where immediate access to doctors and hospitals is uncommon, this application may benefit parents who would not otherwise have easy access to the checkups that pediatricians provide, allowing them access to medical services that they might not otherwise have.

\end{singlespace}

\subsection{Scope}

\begin{singlespace}
\noindent
The application shall be used from the perspective of a household. Parents/guardians will be using it for independent use, whether that entails growth concern or fun surrounding their baby. User feedback has directed us to the idea that keeping track of a baby's growth is an interesting and fun activity, outside of the immediate health concerns to their child. The application will have the capability to inform concerned parents about the growth of their baby if they seek the medical meaning of growth.\\ \par

The scope of this application will not be delving into medical professional use. No regulations in accordance to HIPAA or other medical regulations will be accomplished.
\end{singlespace}

\subsection{Glossary}
\begin{singlespace}

\begin{enumerate}[labelsep=2em,leftmargin=.5in]
    {\item \bfseries Swift: } a general-purpose, multi-paradigm, compiled programming language developed by Apple Inc. for their ecosystem of operating systems, including iOS. It's also supported on Linux and z/OS.
    
    {\item \bfseries Computer vision: } an interdisciplinary field that studies how computers can be programmed to understand digital images and videos at a high level. In other words, it studies how computers can automate tasks done by the human visual system.
    
    {\item \bfseries ARKit: } an augmented reality framework for iOS that provides device motion tracking, camera scene capture, advanced scene processing, and display conveniences to aid the developers in building an AR experience.
    
    {\item \bfseries World Health Organization (WHO): } a specialized agency of the United Nations concerned with international public health, including children's health policies.
\end{enumerate}
\end{singlespace}

%\subsection{Overview}
%\begin{singlespace}
%\noindent
%\end{singlespace}

\section{Overall Description}
\begin{singlespace}
\noindent
The following sections will outline a high-level description of the product. Specifics regarding the product perspective, functions, user characteristics, constraints, assumptions and dependencies, and apportioning of requirements will be covered.
\end{singlespace}

\subsection{Product Perspective}
\begin{singlespace}
\noindent
    The application is for a parent who would like to track their baby's growth and compare it against standard WHO growth charts. Specifically, the parent would be interested in seeing a report, such as a  table or chart, of their baby's growth over time; additionally, they would like to see both the current measurements and growth rate compared against standards to see if their baby was growing healthily.
\end{singlespace}

\subsection{Product Functions}
\begin{singlespace}
\noindent
Users will measure their baby in a similar manner to measuring with a ruler. The application will mimic the ruler by placing two dots on the phone's screen and automatically measuring the distance between the points. Users will see their baby through the camera and be able to interact with the virtual environment that the camera creates. General health information can be read in text form or they can visualize their baby's growth in a comparative line chart that takes in data from the World Health Organization.
\end{singlespace}

\subsection{User Characteristics}
\begin{singlespace}
\noindent
The user will be using this application for one of two purposes: family fun or medical concern/reassurance. The product to some parents/guardians will be purely used because they find joy in finding out how much the baby is growing. Another use is by concerned parents/guardians who think their baby is growing too little or too much. They want information based on the growth of the baby and that's what this application will do. Growing an inch could be a heartwarming moment or it could put a restless mind to ease, which are the feelings the user will have when using this product.
\end{singlespace}

\subsection{Constraints}
\begin{singlespace}
\noindent
Our application is not meant to be a replacement for scientific instruments. In other words, it's not recommended for pediatric use. It's a handy tool for a parent who would like a "good enough" estimate of their baby's height.
\end{singlespace}

\subsection{Assumptions and Dependencies}
\begin{singlespace}
\noindent
ARkit requires an iPhone 6s or above to run the software. The application will be built directly for an iPhone 11 since that is the testing tool we have but should have a responsive design for all iPhones capable of running ARKit.

Core Data is being used as a local database within the iPhone.

Charts library by Daniel Ghindi are being used for displaying a comparative line chart.
\end{singlespace}

\subsection{Apportioning of Requirements}
\begin{singlespace}
\noindent
Khoa Tran will being working on the ARKit measuring portion of the application. Aaron Leenknecht will be working on the user interface and the charting library. Aidan Grimshaw will be working on the local database and the charting library.
\end{singlespace}

\section{Specific Requirements}
\begin{singlespace}
The specific requirements for this application will include the front-end and back-end interfaces for the application. The functionality, performance, usability, compatibility and compliance will have specific features that will be detailed with performance measures that we plan to obtain.
\end{singlespace}
\subsection{Interfaces}
\begin{singlespace}
\noindent
The two separate interfaces: user and software, depict the same application. The user interface is from the front-side, what the users can see and control. The software side is how the application components interact behind the scenes.
\end{singlespace}

\subsubsection{User Interfaces}
\begin{singlespace}
\noindent
The concept behind the user interface is to keep it as clean and simple as possible. Users shall begin by opening the app into a hub screen with five options available to them. At the top will be a create profile button since that is the first thing the user must accomplish when using the application. A picker wheel with all profiles will be available to the user as the next item in the column. The next three items in the column will be matching buttons that go to: take measurement, view comparative chart, and get health information.\\
\end{singlespace}

\subsubsection{Software Interfaces}
\begin{singlespace}
\noindent
The software will have three main components: application base, ARkit and the database. The application base will be the code needed to launch the app through an iOS device along with the code the controls user interaction. The ARkit is a framework that will be incorporated to track the visual representative while obtaining the required measurements. Our database will be to hold the values that the ARkit gathers, so that the application base can then take the values and pull medical information to the user from an online source.\\ \par

The three main components interact cohesively and simultaneously. The application base will be interacting with the users, who are using the ARkit, which in turn uses the database. The database receives measurements from the ARkit and the database sends those measurements to the application base to decide what to send the user.
\end{singlespace}

\subsection{Functional Requirements}
\begin{singlespace}
\noindent
The functional requirements detail the features and functions that the end product must be able to perform.\\

\textbf{ID: FR-0}\\
TITLE: Measure baby height\\
DESC: Height must be measured using the camera on an iPhone to emulate a virtual environment where data calculations can be done.\\
RAT: Core requirement of being able to measure baby growth and natural progress\\
DEPEND: ARKit\\


\textbf{ID: FR-1}\\
TITLE: Store baby measurements\\
DESC: Store all baby measurements in local database under child entities such that each parent can have multiple children which have a timeseries array of height and head width measurements. May use SQLite or some noSQL equivalent. May do an online DB for user backups depending on extra time\\
RAT: Users should be able to compare their baby measurements against previous measurements to track growth\\
DEPEND: Core Data\\

\textbf{ID: FR-2}\\
TITLE: Baby growth charts\\
DESC: The application should have charts that allow the user to see the baby's growth over time, and to see the baby's growth over time overlaid with growth percentiles for a healthy child of the baby's age\\
RAT: Users should be able to check growth in a friendly and intuitive way that allows them to quickly grasp information\\
DEPEND: Charts by Daniel Ghindi\\

\textbf{ID: FR-3}\\
TITLE: Baby Choosing Menu\\
DESC: The user should be able to choose between multiple different children to measure. Each child should have a different series of measurement records associated with it\\
RAT: Parents should be able to measure each of their children and store the results separately if they have multiple children\\
DEPEND: None\\
\end{singlespace}

\subsection{Performance and Usability Requirements}
\begin{singlespace}
\noindent
The performance and usability requirements concern the execution speed, setup, and tutorial  to which the end product must conform.\\

\textbf{ID: PR-0}\\
TITLE: Measurements taken with $>$90\% accuracy\\
DESC: In order to build a useful application that users trust, the application must $>$90\% accurate in the measurements it gathers. If measurements are more than 10\% off, growth charts may be distorted by errors and the system might incorrectly place the baby in an unhealthy growth percentile\\
RAT: Accuracy is a performance requirement that the application depends on\\
DEPEND: None\\

\textbf{ID: PR-1}\\
TITLE: Measurements taken in $<$10 seconds\\
DESC: The application should be easy to use, requiring few user interactions to get the measurements it needs.\\
RAT: If the user experience is overly onerous, users will default back to manual measurements using a tape measure or similar\\
DEPEND: None\\


\end{singlespace}



\vfill
\noindent\begin{tabular}{ll}
\makebox[3.5in]{\hrulefill} & \makebox[1.5in]{\hrulefill}\\
Client Signature & Date\\
[4ex]% adds space between the two sets of signatures
\makebox[3.5in]{\hrulefill} & \makebox[1.5in]{\hrulefill}\\
Group Signature & Date\\
[4ex]% adds space between the two sets of signatures
\makebox[3.5in]{\hrulefill} & \makebox[1.5in]{\hrulefill}\\
Group Signature & Date\\
[4ex]% adds space between the two sets of signatures
\makebox[3.5in]{\hrulefill} & \makebox[1.5in]{\hrulefill}\\
Group Signature & Date\\
\end{tabular}

\bibliographystyle{IEEEtran}
\bibliography{requirements/req.bib}

\end{document}

